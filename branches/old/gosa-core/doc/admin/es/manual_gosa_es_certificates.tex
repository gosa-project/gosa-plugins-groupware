\chapter{Seguridad y Certificados}
\section{Introduction SSL}
\section{Creaci�n de certificados}
La seguridad es uno de los puntos mas importantes al configurar un servidor, necesitaremos un entorno seguro donde no permitir que los usuarios manipulen y accedan a codigo o programas.

Una formas de conseguir esto es usando encriptaci�n, con lo que buscamos que los usuarios y el servidor se comuniquen de forma que nadie mas pueda acceder a los datos. Esto se consigue con encriptaci�n.

La otra manera de asegurar el sistema es que si existe alg�n fallo en el sistema o en el c�digo, y un intruso intenta ejecutar codigo, este se vea incapacitado, ya que existen poderosas limitaciones, como no permitir que ejecute comandos, lea el codigo de otros script, no pueda modificar nada y tenga un usuario con muy limitados recursos.
\subsection{ Certificados SSL}
\label{down_ssl}
\noindent Necesitaremos \textbf{openSSL}, existe en todas las distribuciones y tiene documentaci�n en su pagina web\cite{ssldoc}.

\noindent Las fuentes se pueden descargar de \hlink{http://www.openssl.org/source/}

\noindent Existe amplia documentaci�n sobre encriptaci�n y concretamente sobre SSL, un sistema de encriptaci�n con clave publica y privada.

\noindent Como el paquete openSSL ya lo tenemos instalado a partir de los pasos anteriores, debemos crear los certificados que usaremos en nuestro servidor web.

\noindent Supongamos que guardamos el certificado en /etc/apache2/ssl/gosa.pem

\jump
\begin{rbox}[label=Pem Certificate]
# FILE=/ect/apache2/ssl/gosa.pem
# export RANDFILE=/dev/random
# openssl req -new -x509 -nodes -out $FILE -keyout /etc/apache2/ssl/apache.pem
# chmod 600 $FILE
# ln -sf $FILE /etc/apache2/ssl/`/usr/bin/openssl x509 -noout -hash < $FILE`.0
\end{rbox}
\jump


\noindent Con esto hemos creado un certificado que nos permite el acceso SSL a nuestras p�ginas.

\noindent Si lo que queremos es una configuraci�n que nos permita no solo que el tr�fico est� encriptado, sino que adem�s el cliente garantice que es un usuario v�lido, debemos provocar que el servidor pida una certificaci�n de cliente. 
\newpage
\noindent En este caso seguiremos un procedimiento mas largo, primero la creaci�n de una certificaci�n de CA:
\jump
\begin{rbox}
# CAFILE=gosa.ca
# KEY=gosa.key
# REQFILE=gosa.req
# CERTFILE=gosa.cert
# DAYS=2048
# OUTDIR=.
# export RANDFILE=/dev/random
# openssl req -new -x509 -keyout $KEY -out $CAFILE -days $DAYS
\end{rbox}
\jump

Despu�s de varias cuestiones tendremos una CA, ahora hacemos un requerimiento para un nuevo certificado:
\jump
\begin{rbox}
# >DAYS=365
# >openssl req -new -keyout $REQFILE -out $REQFILE -days$DAYS
\end{rbox}
\jump

Creamos una configuraci�n para usar la CA con openssl y la guardamos en openssl.cnf:
\jump
\begin{rbox}
HOME = .
RANDFILE = $ENV::HOME/.rnd
[ ca ]
default_ca  = CA_default
[ CA_default ]
dir = .
database = index.txt
serial = serial
default_days = 365
default_crl_days= 30
default_md = md5
preserve = no
policy = policy_anything
[ policy_anything ]
countryName = optional
stateOrProvinceName  = optional
localityName = optional
organizationName = optional
organizationalUnitName  = optional
commonName = supplied
emailAddress = optional
\end{rbox}
\jump

Firmamos el nuevo certificado:
\jump
\begin{rbox}
# touch index.txt
# touch index.txt.attr
# echo "01" >serial
# openssl ca -config openssl.cnf -policy policy_anything \
    -keyfile $KEY -cert $CAFILE -outdir . -out $CERFILE -infiles $REQFILE
\end{rbox}
\jump

Y creamos un pkcs12 para configurar la certificaci�n en los clientes:
\jump
\begin{rbox}
# openssl pkcs12 -export -inkey $KEY -in $CERTFILE -out certificado_cliente.pkcs12
\end{rbox}
\jump
Este certificado se puede instalar en el cliente, y en el servidor mediante la configuraci�n explicada en cada uno, esto nos dar� la seguridad de que su comunicaci�n ser� estrictamente confidencial.\\
